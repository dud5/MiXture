%%%%%%%%%%%%%%% Updated by MR March 2007 %%%%%%%%%%%%%%%%
\documentclass[12pt,a4paper]{article} \usepackage{a4wide}
\bibliographystyle{plain}
\newcommand{\fs}{F\nolinebreak\hspace{-.05em}\raisebox{.6ex}{\tiny\bf
    \#}}

\parindent 0pt
\parskip 6pt

\begin{document}

\thispagestyle{empty}

\rightline{\large Eduardo Munoz} \medskip \rightline{\large Magdalene
  College} \medskip \rightline{\large em487}

\vfil

\centerline{\large Computer Science Tripos Part II Project Proposal \texttt{draft}}
\vspace{0.4in} \centerline{\Large\bf Multilanguage programming in
  \fs\ and JavaScript} \vspace{0.3in} \centerline{\large \emph{\today}}

\vfil

{\bf Project Originator:} \emph{Eduardo Munoz}

\vspace{0.5in}

{\bf Project Supervisor:} \emph{Tomas Petricek}

\vspace{0.2in}

{\bf Signature:}

\vspace{0.5in}

{\bf Director of Studies:} \emph{Dr John Fawcett}

\vspace{0.2in}

{\bf Signature:}

\vspace{0.5in}

{\bf Overseers:} \emph{Dr S. Teufel} and \emph{Dr J. Crowcroft}

\vspace{0.2in}

{\bf Signatures:}

\vfil \eject


\section{Introduction and description of the work}

Selecting the programming language to use when implementing a certain
algorithm is crucial in the success of the project. For this reason,
large scale software systems tend to be written in several
languages. However, the interactions between those components may be
an issue.  There are several approaches to tackle this:

\begin{enumerate}

\item Foreign function interfaces (FFI): mechanism by which a program
  written in one language may call procedures from a program in a
  different language.

\item Multilanguage runtimes: Several programming languages target the
  same architecture, allowing a richer interaction between languages:
  e.g. inherit classes in one language defined in another. Examples of
  multilanguage runtimes are Java, Scala, Jython, etc. targeting the
  Java Virtual Machine; all programs written for the .NET framework
  target the CLR.

\item Embedded interpreters: implementing an interpreter of the target
  language in the host language and use a type-indexed embedding and
  projection algorithm.

\end{enumerate}

In this project, we intend to explore the ideas described in
\cite{journals/toplas/MatthewsF09}, for an ML-like language and
JavaScript. This includes producing an implementation of the
\emph{lump embedding} and the \emph{natural embedding}. The former
concept is related to FFI frameworks (such as the Java Native
Interface), where each environment sees foreign values as ``lumps''
(values with an opaque type). The natural embedding provides a richer
interoperability between the two languages, allowing values to be
converted across the boundary from one language to the other. The
implementation for the natural embedding can be seen as FFI with some
properties of multilanguage runtimes.

An example of using the lump embedding (note that the syntax hasn't
been decided yet, this is for illustrative purposes):

\begin{verbatim}
    (* implementation missing; types subject to change *)
    let eval_js x : JS list->JS = raise notImplemented;
    let apply_lump f x = eval_js([f,x]);
    let succ:Lump = JS("function(a) {return a+1;}");
    in apply_lump(succ, JS(“3”));

    > val it: JS(“4”) : Lump
\end{verbatim}

In this case, the ML-like language can only interact with values of
type \texttt{Lump} to interoperate with JavaScript (using the type constructor
\texttt{JS}).

An example of using the natural embedding (note that the syntax hasn't
been decided yet, this is for illustrative purposes):

\begin{verbatim}
    let test (x:string->unit) :unit = x("testing");
    in test(JS(string, unit, "function(s) {print(s)}"));
    > val it: () : unit
\end{verbatim}


\section{Starting point}

This project will involve material from \emph{Types} (deal with
language boundaries), \emph{Semantics of Programming Languages}
(specify the behavior of the framework), \emph{Compiler Construction}
(analyze the runtime of the JavaScript engine to integrate it) and
\emph{Foundations of Computer Science} (functional programming, ML).


\section{Resources required}

The implementation will be completed on my own laptop.

I will make use of a JavaScript engine (e.g. \emph{V8}) and an ML-like
functional programming language (OCaml or \fs).


\section{Substance and structure of the project}

The goal of this project is to design and implement a framework for the multilanguage programming paradigm. Some aspects in which two programming languages can differ in are:

\begin{enumerate}\interlinepenalty10000
\item \emph{Type system}: JavaScript is an untyped\footnote{There is some
    confusion with the terms \emph{dynamically typed} and
    \emph{untyped}. In the academic literature, the term dynamically typed
    was introduced much later than untyped to mean the same
    concept. Perhaps the best categorization for JavaScript is \emph{weakly dynamically typed}.}
  language, while \fs\ has strong static typing with type inference.
\item \emph{Evaluation rules}: for instance, JavaScript allows the evaluation of two empty lists \cite{ECMA-262}; OCaml however forbids this operation because it doesn't type check:
\begin{verbatim}
(JavaScript)
js> [] + []

// (empty string)

--------------------
(OCaml)
# [] + [];;
Error: This expression has type 'a list
       but an expression was expected of type int
\end{verbatim}

\item \emph{Values}: for instance, JavaScript treats all numbers as floating point numbers:
\begin{verbatim}
(JavaScript)
js> 1000000000000000000 + 1
1000000000000000000
js> 4611686018427387903 + 1
4611686018427388000
--------------------
(OCaml)
# 1000000000000000000 + 1;;
- : int = 1000000000000000001
# 4611686018427387903 + 1;;
- : int = -4611686018427387904
\end{verbatim}
\item \emph{Syntax}: JavaScript's syntax is loosely based on that by C and Java, whereas the syntax of \fs\ is that of the ML-language family.
\item \emph{Purpose}: JavaScript is the language of the web, used mainly to enhance user interfaces and dynamic websites. \fs\ is a general purpose language.

\end{enumerate}

Work will have to be carried out in the following areas:
\begin{enumerate}

\item \emph{Type system}: handle types at the boundaries of the programming languages,
  making sure the type soundness of \fs\ is preserved (possible use of
  contracts).

\item \emph{Evaluation rules}: generate glue / wrapper code that allow treating a JavaScript function as \fs\ function (which might need contracts).

\item \emph{Values}: access JavaScript values from the \fs\ runtime using the
  JavaScript engine.

\item \emph{Syntax}: define a clear interface with which JavaScript code will be
  embedded in \fs\ source files (either inline or loading a file), both
  for the lump and the natural embedding.


\end{enumerate}

\section{Possible extensions}
\begin{enumerate}
\item Syntax-check the JavaScript syntax using \fs\ type providers.
\end{enumerate}

\section{Evaluation strategy}

Quantitative:
\begin{itemize}
\item Compare the performance difference between the system
  implementing the lump embedding and the natural embedding.
\item Estimate the relationship between the number of “foreign”
  boundary crossings and the execution time.
\item Compare the performance difference with other systems that allow
  some interoperability between \fs\ and JavaScript.
\end{itemize}

Qualitative:
\begin{itemize}
\item Show the expressiveness of the system.
\item Proof of correctness, by executing some tests.
\end{itemize}


\section{Backup strategy}

All source files (code and \LaTeX) are in my
local machine in a \emph{git} repository, which is hosted at
\emph{Bitbucket} and also replicated to an external hard drive.  The
git repository will be useful when writing the dissertation as it will
be used as a work log.

\section{Success criteria}

\begin{enumerate}

\item The resulting framework should not take a significant amount of
  time than executing the respective monolingual runtimes.

\item The lump embedding implementation should be able to pass values
  from JavaScript to \fs\ and then pass them back.

\item The natural embedding should be able to pass a function from
  JavaScript to \fs\ (and the other way round) and invoke it on the
  other side of the boundary

\item A convenient syntax for multilanguage programming has been
  designed.

\item Tests of correctness have been passed.

\end{enumerate}

\section{Timetable and milestones}

\subsection*{Initial preparation: 18.10.2012 - 31.10.2012}

Install \fs\ on my laptop (using the Mono framework for .NET); install
Windows as a fallback.  Revise ML and become familiar with the
differences in \fs.  Keep reading research papers and articles about
the subject.  Research JavaScript engines and decide which one offers
the best API to access JavaScript values.
\\{\bf Milestone}:

\subsection*{01.10.2012 - 14.11.2012 }
Decide on a specific syntax for embedding JavaScript inside \fs.
\\{\bf Milestone}:

\subsection*{15.11.2012 - 28.11.2012}
A
\\{\bf Milestone}:

\subsection*{29.11.2012 - 12.12.2012}
B
\\{\bf Milestone}:

\subsection*{13.12.2012 - 26.12.2012}
C
\\{\bf Milestone}:

\subsection*{27.12.2012 - 09.01.2012}
D
\\{\bf Milestone}:

\subsection*{10.01.2012 - 23.01.2012}
E
\\{\bf Milestone}:

\subsection*{24.01.2012 - 06.02.2012}
Write the introduction, preparation and implementation sections.
\\{\bf Milestone}: the dissertation document has been started.

\subsection*{07.02.2012 - 20.02.2012}
Finish off the following sections: preparation and implementation.
Take evaluation metrics.
\\{\bf Milestone}: preparation and implementation are finished. Evaluation data is gathered.

\subsection*{21.02.2012 - 06.03.2012}
Write the evaluation and conclusion sections of the dissertation.
\\{\bf Milestone}: a draft of the whole dissertation is now complete.

\subsection*{07.03.2012 - 20.03.2012}
Make formatting changes to the document. Send final draft to my supervisor and make small adjustments if suggested.
\\{\bf Milestone}: the document’s formatting is final.
\subsection*{21.03.2012 - 03.04.2012}
Make any last-minute minor changes to the dissertation / code if suggested by my supervisor or DoS.
\\{\bf Milestone}: have the dissertation approved, hand it in.

\bibliography{bibliography}
\end{document}
